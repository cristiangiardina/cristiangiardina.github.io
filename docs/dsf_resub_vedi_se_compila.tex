\documentclass[vecphys]{svmult}


% choose options for [] as required from the list
% in the Reference Guide, Sect. 2.2

\usepackage{makeidx}         % allows index generation
\usepackage{graphicx}        % standard LaTeX graphics tool
                             % when including figure files
\usepackage{multicol}        % used for the two-column index
\usepackage[bottom]{footmisc}% places footnotes at page bottom
% etc.
% see the list of further useful packages
% in the Reference Guide, Sects. 2.3, 3.1-3.3

\makeindex             % used for the subject index
                       % please use the style sprmidx.sty with
                       % your makeindex program


%%%%%%%%%%%%%%%%%%%%%%%%%%%%%%%%%%%%%%%%%%%%%%%%%%%%%%%%%%%%%%%%%%%%%


% \documentclass[10pt]{article}
% \setlength{\textwidth}{27pc}
% \setlength{\textheight}{43pc}
% \usepackage{latexsym}
 \usepackage{amssymb}
 \usepackage{amsfonts}
 \usepackage{amsmath}
%\usepackage{theorem}

%\theorembodyfont{\upshape}


%\newtheorem{theorem}{Theorem}
%\newtheorem{lemma}{Lemma}
%\newtheorem{corollary}{Corollary}
%\newtheorem{definition}{Definition}
%\newtheorem{remark}{Remark}
%\newtheorem{proposition}{Proposition} 

%\renewcommand{\baselinestretch}{1.4}

\begin{document}


\title*{Self-Averaging Identities \\
for Random Spin Systems}
\author{Luca De Sanctis\inst{1}
and Silvio Franz\inst{2}}
\institute{
ICTP, Strada Costiera 11, 34014 Trieste, Italy,
\texttt{lde\_sanc@ictp.it} ,
\and 
LPTMS, b\^atiment 100,
Universit\'e Paris-Sud 11,
Centre scientifique d'Orsay,
15 rue Georges Cl\'emenceau,
91405 Orsay cedex
\texttt{silvio.franz@lptms.u-psud.fr}}


\maketitle

\begin{abstract}
We provide a systematic treatment of self-averaging identities,
whose validity is proven in integral average, for dilute spin glasses.
The method is quite
general, and as a special case recovers the Ghirlanda-Guerra 
identities,
which are therefore proven, together with their extension, 
to be valid in dilute spin
glasses. We focus on dilute spin glasses, but the results hold
in all models enjoying stability with respect the perturbations we
introduce; although such a stability is believed to hold for several
models, we do not classify them here.
\end{abstract}

\noindent{\em Key words and phrases:} diluted 
spin glasses, Ghirlanda-Guerra, self-averaging.

\noindent{\em Mathematics Subject Classification:} 82B44; 82B20




%%%%%%%%%%%%%%%%%%%%%%%%%%%%%%%%%%%%%%%%%%%%%%%%%%%%%%%%%%%%%%%%%%%

\section{Introduction}


Despite many years of intense work, and the much awaited proof of the
validity of the Parisi ansatz for the free-energy of the 
Sherrington-Kirkpatrick (SK) and
related models, the mathematical comprehension of the thermodynamics of
mean field spin glasses remains largely incomplete. We know from
theoretical physics that in fully connected models all the properites
of the low temperature spin glass phase can be encoded in the
probability distribution of the overlap between two different copies of
the system. The analysis of Parisi et al. predicts an ultrametric
organization of the phases (see \cite{mpv} and references therein).
So far the rigorous proof (or disproof) of ultrametricity, and, more in
general, the analysis of the structure of Gibbs measures at low
temperature, turned out to be a very difficult task. A step in this
direction was performed by Ghirlanda and Guerra in \cite{gg}. They found
a simple and elegant way, based on the self-averaging of the
internal energy, to prove a remarkable property of the overlaps. Given
$s$ replicas, the Gibbs measure must be such that
when one adds a further replica this is either
identical to one these, or statistically independent of them; each
case occurring with the same probability. More
generally, various constraints on the distribution of the different
overlaps have been found in the same spirit \cite{ac, parisi}. Such
features have found several applications \cite{talabook, bovierbook}
in the rigorous analysis of spin glass models. For example, the
property of non-negativity of the overlap, which in some models plays
a role in turning the cavity free-energy into a rigorous lower bound,
turns out to be a consequence of the Ghirlanda-Guerra self-averaging
identities \cite{talabook}. In the same way these identities have
a role in the rigorous analysis of spin glasses close to the critical
temperature \cite{abds}.

In more general spin-glass systems, like finite dimensional systems or
spin systems on random graphs, the statistics of the overlap is not
enough to fully characterize the low temperature spin glass phase. For
instance, in diluted models the statistics of the local cavity
fields, or equivalently of all the multi-overlaps, is necessary to
describe the low temperature thermodynamic properties. In this
paper, we analyse two families of identities for the local fields and
multi-overlap distributions that are a consequence of self-averaging
relations. We will see that first of the two families is
a consequence of the self-averaging with respect to the 
Gibbs measure or, equivalently, 
of stochastic stability, as the two phenomena turn out to be 
equivalent. The other family of identities is instead a consequence 
of self-averaging with respect to the global measure
(quenched after Gibbs). The second family contains the first.
The form of the identities we obtain
is due only to the form of the perturbations we introduce,
and does not depend on the specific form of the Hamiltonian.
However, the self-averaging at the basis of the results does
not necessarily holds for all Hamiltonians. So we will 
stick with the case of dilute spin glass (Viana-Bray model),
for which the self-averaging is assured, but our method shows that
the same identities we 
find hold whenever the self-averaging is true. We do not
provide a classification of all the models exhibiting 
self-averaging when perturbed.  
A second reason to use the example of spin
models on sparse random graphs (dilute spin glass models), is that we
expect that our results could provide hints for progresses in the mathematical
analysis of the low temperature phases. Diluted mean field spin
glasses have, in recent time, attracted a lot of attention in
statistical physics, due to the intrinsic interest of spin glasses
where each spin interacts with a finite number of variables, but more
importantly because fondamental problems in computer science, such as
the random K-SAT and graph coloring, the random X-OR-SAT, tree
reconstruction \cite{mm1} and others, admit a formulation in
terms of spin glass systems on random graphs. 
The cavity approach to these problems has led in
many cases to results believed to be exact, albeit for the moment
several rigorous proofs are still lacking. 

One of the two families of the identities that we will find appeared already in
\cite{flt} to discuss free energy bounds in diluted models with
non-Poissonian connectivity. In \cite{flt} such a family of identities
was shown to be a consequence of self-averaging of certain
random polynomial function of some spins variables,
and the self-averaging was deduced from the convexity of the 
perturbed free energy. 
Here we re-derive this family of identities with a different
strategy, employing stochastic stability
of the free energy
with respect to suitable perturbations of the Hamiltonian,
and we show that the stability implies self-averaging. 
The use of stochastic stability makes our results valid
even when a measure different from the Gibbs one is considered,
provided certain conditions hold (we will briefly hint at this
when introducing Random Multi-Overlap Structures).
Moreover, our method shows that the constraints we find
are valid whenever stochastic stability or self-averaging hold,
whichever turns out to be easier to study, and in the physics literature 
many models have been investigated from at least one of the two
points of view.
We also exhibit a second family of new identities, which contains
the first family and follows from the self-averaging with respect
to the quenched expectation.

For both families, we will start with the two-spin model, perturbed with
a two-spin random field. This provides identities involving squared
overlaps. The same identical method, reproduced for generic $p$-spin
interactions, yields the same identities involving the $p$-th power of the
overlaps. Therefore considering Hamiltonians and perturbations with all $p$-spin
interactions we will conclude that the identities hold for all regular functions
of the overlaps.

Let us stress here that the basic tool we employ is the introduction of suitable 
perturbations, such that the pressure (minus the free energy divided by the temperature) 
is convex in all the perturbing parameters.
This guarantees the existence of the derivatives with respect to the perturbing parameters
only almost everywhere. Therefore the relations we find are valid only when one 
integrates (with Lebesgue measure) back against the perturbing parameters
over any given intervals. We summarize this by recalling that our relations hold
in ``integral average''.

%%%%%%%%%%%%%%%%%%%%%%%%%%%%%%%%%%%%%%%


\section{The notations}

We will deal with the stereotypical dilute spin glass model,
the Viana-Bray (VB), for which we are about to describe the notations we need
to derive our results in the next two sections.
Let
$\alpha, \beta$ be 
non-negative real numbers
(degree of connectivity and inverse temperature
respectively);
$P_\zeta$ be a Poisson random variable of mean $\zeta$;
$\{i_\nu\}, \{j_\nu\}$, etc. be independent identically 
distributed random variables, 
uniformly distributed over the points $\{1,\ldots, N\}$;
$\{J_\nu\}, J$, etc. be
independent identically distributed 
random variables, with symmetric distribution;
$\mathcal{J}$ be the set of all the quenched 
random variables  above. The map
$\sigma: i \rightarrow \sigma_{i},\ i\in\{1,\ldots , N\}$ 
is a spin configuration from the 
configuration space $\Sigma=\{-1,1\}^{N}$;
$\pi_{\zeta}(\cdot)$ is the Poisson measure
of mean $\zeta$.
The Hamiltonian of the Viana-Bray model is defined as
\begin{equation}
\label{ham}
H_N(\sigma, \alpha; \mathcal{J})=
-\sum_{\nu=1}^{P_{\alpha N}} J_\nu \sigma_{i_\nu}\sigma_{j_\nu}\ .
\end{equation}
We will limit to the case $J=\pm 1$, without loss of 
generality \cite{gt1}.
We follow the usual basic definitions and notations 
of thermodynamics for the partition function $Z_{N}$,
the pressure $p_{N}$,
the free energy per site $f_{N}$ and its thermodynamic limit $f$,
so to have in general
\begin{equation}
Z_{N}(\beta,\alpha)\equiv Z(H_{N}; \beta,\alpha; \mathcal{J})=\sum_{\{\sigma\}}
\exp(-\beta H_N(\sigma,\alpha; \mathcal{J}))\ ,
\end{equation}
\begin{equation}
p_{N}(\beta,\alpha)=-\beta f_N(\beta,\alpha)=\frac1N \mathbb{E}
\ln Z_N(\beta,\alpha)\ ,\ f(\beta,\alpha)=\lim_{N\to\infty} f_N(\beta,\alpha)\ .
\end{equation}
The Boltzmann-Gibbs average of an observable 
$\mathcal{O}:\Sigma\to\mathbb{R}$ is
\begin{equation}
\Omega(\mathcal{O})=
Z_N(\beta,\alpha; \mathcal{J})^{-1}
\sum_{\{\sigma\}}\mathcal{O}(\sigma)\exp(-\beta
H_N(\sigma,\alpha; \mathcal{J}))\ ,
\end{equation}
and $\langle\mathcal{O}\rangle=\mathbb{E}\Omega(\mathcal{O})$ is the
global average, where
$\mathbb{E}$ denotes the average with respect to the quenched variables.
When dealing with more than one configuration, we need the 
product measure of the needed copies of $\Omega$, which will be denoted
again by $\Omega$.

The multi-overlaps $q_{1\cdots m}:\Sigma^{m}\to[-1,1]$,
where we use the notation 
$\Sigma^{n}=\Sigma^{(1)}\times\cdots\times\Sigma^{(n)}$, 
among the ``replicas'' 
$\Sigma^{(r_{1})}\ni\sigma^{(r_{1}(},\ldots,\Sigma^{(r_{n})}\ni\sigma^{(r_{n})}$ 
is defined by
\begin{equation}
\label{overlap}
q_{r_1\cdots r_n}=\frac{1}{N}\sum_{i=1}^N\sigma_i^{(r_1)}
\cdots\sigma_i^{(r_n)}\ ,
\end{equation}
but sometimes we will just write $q_n$; $q_{1}$ can be identified with
the magnetization $m$
$$
m=\frac{1}{N}\sum_{i=1}^N\sigma_i\ .
$$
%Notice that $\sigma^{s}_{i}$ is the $i$-th spin 
%from the replica $s$, $\Sigma^{(s)}$, not the $s$-th power of 
%$\sigma_{i}$, i.e.
%$s$ is an index and not an exponent.

Notice that
\begin{equation}
\label{q}
\mathbb{E}[\Omega
(\sigma_{i_{1}})]^{2n}=\langle q_{1\cdots 2n}\rangle\ ,\ 
\mathbb{E}\Omega
(\sigma_{i_{1}})=\mathbb{E}\Omega
(m)=\langle m \rangle\ ,
\end{equation}
and that
\begin{equation}
%\label{qbis}
\mathbb{E}[\Omega
(\sigma_{i_{1}}\cdots\sigma_{i_{p}})]^{2n}=\langle q^{p}_{1\cdots 2n}\rangle\ ,\ 
\mathbb{E}\Omega
(\sigma_{i_{1}}\cdots\sigma_{i_{p}})=\mathbb{E}\Omega
(m^{p})=\langle m^{p} \rangle\ ,
\end{equation}
for all integer $n$ and $p$. 
%Notice that in the case of the overlaps,
%$p$ are not apices labeling replicas but they are just exponents,
%contrarily to the notation for the spins.


%%%%%%%%%%%%%%%%%%%%%%%%%%%%%%%%%%%%%%%
%%%%%%%%%%%%%%%%%%%%%%%%%%%%%%%%%%%%%%%
\section{Stochastic Stability and self-averaging of the Gibbs measure}
\label{qs}

In the study of finite connectivity models it emerged that
in a suitable propability space it is possible to
formulate an exact variational principle for the computation
of the free energy. This was obtained with the introduction
of Random Multi-Overlap Structures (RaMOSt). We refer
to \cite{lds1} for details.
The ROSt approach is based on the use of generic random weights
to average the ``cavity'' part and the relative ``internal
correction'' in the free energy (these are the numerator and
the denominator of the trial free energy $G_N$ introduced in
(\ref{gtfdilute}). See \cite{lds1} for details).
Here we are not interested in a detailed discussion of
the RaMOSt approach, but we study the effect of a 
perturbation to the measure of our model, which does not
need to be the Gibbs measure. That is why introduce this
more general weighting scheme, although the reader may
keep in mind the Gibbs measure as a guiding example.

\subsection{Random Multi-Overlap Structures}

The proper framework for the calculation of the free energy per spin
is that of the Random Multi-Overlap Structures (RaMOSt, see
\cite{lds1} for more details).
\begin{definition}
  Given a probability space $\{\Omega,\mu(d\omega)\}$,
  a {\bf Random Multi-Overlap Structure}
  $\mathcal{R}$ is a triple
  $(\tilde{\Sigma}, \{\tilde{q}_{2n}\}, \xi)$ where
  \begin{itemize}
  \item $\tilde{\Sigma}$ is a discrete space;
  \item $\xi: \tilde{\Sigma}\rightarrow\mathbb{R}_+$
    is a system of random weights, such that $\sum_{\gamma\in\tilde{\Sigma}}
    \xi_\gamma\leq\infty$ $\mu$-almost surely;
  \item $\tilde{q}_{2n}:\tilde{\Sigma}^{2n}\rightarrow\mathbb{R},
    n\in\mathbb{N}$ is a positive semi-definite \emph{Multi-Overlap
      Kernel} (equal to 1 on the diagonal of $\tilde{\Sigma}^{2n}$, so
    that by Schwartz inequality $|\tilde{q}|\leq 1$).
  \end{itemize}
\end{definition}
A RaMOSt needs to be equipped with $N$ independent copies
of a random field
$\{\tilde{h}^{i}_{\gamma}(\alpha; \tilde{\mathcal{J}})\}_{i=1}^{N}$ and
with another random field
$\hat{H}_{\gamma}(\alpha N; \mathcal{J}^{\prime})$
such that
\begin{eqnarray}
&&\!\!\!\!\frac{d}{d\alpha}\mathbb{E}\ln \sum_{\gamma\in\tilde{\Sigma}}\xi_{\gamma}
\exp(-\beta\tilde{h}^{i}_{\gamma}(\alpha; \tilde{\mathcal{J}}))
 =  2\sum_{n>0}\frac{1}{2n}\tanh^{2n}(\beta)
(1-\langle \tilde{q}_{2n}\rangle) , \label{eta} \\
&&\!\!\!\!\frac{d}{d\alpha}\mathbb{E}\ln \sum_{\gamma\in\tilde{\Sigma}}\xi_{\gamma}
\exp(-\beta \hat{H}_{\gamma}(\alpha N; \mathcal{J}^{\prime}))
= \sum_{n>0}\frac{1}{2n}\tanh^{2n}(\beta)
(1-\langle\tilde{q}^{2}_{2n}\rangle) .\label{kappa}
\end{eqnarray}
The quenched variables in $\tilde{h}$ and $\hat{H}$
are independent one another and independent of
those in the weights $\xi$.
These two fields just introduced are employed in the definition of the
trial pressure
\begin{equation}\label{gtfdilute}
G_{N}(\mathcal{R};\beta,\alpha)=\frac 1N\mathbb{E}\ln
\frac{\sum_{\gamma, \sigma}\xi_{\gamma}
\exp(-\beta\sum_{i=1}^{N}\tilde{h}^{i}_{\gamma}(\alpha; \tilde{\mathcal{J}})
\sigma_{i})}{\sum_{\gamma}\xi_{\gamma}
\exp(-\beta \hat{H}_{\gamma}(\alpha N; \mathcal{J}^{\prime}))}\ .
\end{equation}
Notice that the expectation $\langle\cdot\rangle$ here is not necessarily
the quenched Gibbs one: it is the generic one of the RaMOSt.

The reason why this is the proper framework for the 
calculation of the free energy is explained 
by the next \cite{lds1}
\begin{theorem}[Extended Variational Principle]
Taking the infimum for each $N$
separately of the trial function $G_N(\mathcal{R};\beta,\alpha)$ over the space
of all RaMOSt's, the resulting sequence tends to the limiting pressure
$-\beta f(\beta,\alpha)$ of the VB model as $N$ tends to infinity:
\begin{equation}
\label{evp-d}
-\beta f(\beta,\alpha)=\lim_{N\rightarrow\infty}
\inf_{\mathcal{R}}G_{N}(\mathcal{R};\beta,\alpha)\ .
\end{equation}
\end{theorem}
A RaMOSt $\mathcal{R}$ is said to be optimal if
$G(\mathcal{R};\beta,\alpha)=-\beta f(\beta,\alpha)\ \ \forall\ \beta,\alpha$.
Recall that we denote by $\Omega$ the measure
associated to the RaMOSt weights $\xi$.

The Boltzmann RaMOSt \cite{lds1} is optimal, and
constructed by thinking of a reservoir of $M$ spins $\tau$
$$
\Sigma=\{-1,1\}^M\ni\tau\ ,\ \xi_\tau=\exp(-\beta H_M(\tau,\alpha,\mathcal{J}))\ ,\
$$
$$
\tilde{q}_{1\cdots 2n}=\frac1M\sum_{k=1}^M\tau^{(1)}_k\cdots\tau^{(2n)}_k
$$
with
$$
\tilde{h}^{i}_{\tau}(\alpha,\tilde{\mathcal{J}})=\sum_{\nu=1}^{\tilde{P}_{2\alpha}}
\tilde{J}_{\nu}^{i}\tau_{k_{\nu}^{i}}\ , \
\hat{H}_{\tau}(\alpha N, \mathcal{J}^{\prime})=
-\sum_{\nu=1}^{\hat{P}_{\alpha N}}
\hat{J}_\nu \tau_{k_\nu}\tau_{l_\nu}\ ,
$$
with $\tilde{J},\hat{J}$ all independent copies of $J$, and 
with the random site variables $k_{\nu},l_{\nu}$ all mutually independent and
uniformly distributed over $\{1,\ldots M\}$.

Let $c_{i}=2\cosh(\beta\tilde{h}^{i}_{\tau}(\alpha,\tilde{\mathcal{J}}))$.
It is possible to show \cite{lds1} that
optimal RaMOSt's enjoy the same
factorization property enjoyed by the Boltzmann
RaMOSt and described in the next \cite{lds1}
\begin{theorem}[Factorization of optimal RaMOSt's]
\label{lisboa-d}
The following Ces{\`a}ro  limit
is linear in $N$ and $\bar{\alpha}$
\begin{equation}\label{limrost}
\mathbf{C}\lim_{M}\mathbb{E}\ln\Omega_M
\{c_{1}\cdots c_{N}\exp[-\beta \hat{H}_\tau(\bar{\alpha},\mathcal{J}^{\prime})]\}
=N(-\beta f +\alpha A)+\bar{\alpha}A\ ,
\end{equation}
where
\begin{equation}\label{defa}
A=\sum_{n=1}^{\infty}\frac{1}{2n}
\mathbb{E}\tanh^{2n}(\beta J)(1-\langle q_{2n}^{2}\rangle)\ ,
\end{equation}
and the equality holds in integral average, i.e. once both
sides are integrated against $\alpha$ over any given interval.
\end{theorem}
This factorization property is called {\sl invariance with 
respect to the cavity step},
or {\sl Quasi-Stationarity},
and it is found in the hierarchical Parisi ansatz as well.
When $\bar{\alpha}$ is zero, the theorem above states
the factorization of the cavity fields, and it is possible to show that
from this property one can deduce the family of identities we will
discuss in the next subsection \cite{bds2}. When one removes instead
the cavity terms $c_{1},\ldots , c_{N}$ from the previous theorem,
the statement becomes what is usually referred to as Stochastic
Stability. We will show that the latter too implies the same family
of identities. We will have in mind the case of a small perturbation
of our spin system, but what we find holds for more
general RaMOSt's, provided the previous theorem holds, that is
for Quasi-Stationary RaMOSt's.


%%%%%%%%%%%%%%%%%%%%%%%%%%%%%%%%%%%%%%
%%%%%%%%%%%%%%%%%%%%%%%%%%%%%%%%%%%%%%
\subsection{The first family of identities}

We will now prove a lemma that expresses the stability
of the Gibbs measure of our model against a macroscopic but
small stochastic perturbation. In different terms, the lemma expresses
the linear response of the free energy to the connectivity shift
the perturbation consists of.
The lemma we are about to prove will be used to show that
from stochastic stability one can deduce a certain self-averaging
which in turn imposes a family of constraints on the distribution of the
overlaps.
\begin{lemma}\label{lemma}
  Let $\Omega\ ,\ \langle\cdot\rangle$ be the usual infinite volume
  Gibbs and quenched
  Gibbs expectations at inverse temperature $\beta$, associated with
  the Hamiltonian $H_{N}(\sigma, \alpha; \mathcal{J})$, $N\to\infty$. 
  Then the following equality 
  (understood to be in the thermodynamic limit) holds 
\begin{equation}\label{stability}
\mathbb{E}\ln\Omega\exp\bigg(\beta^{\prime}
\sum_{\nu=1}^{P_{\alpha^{\prime}}}
J^{\prime}_{\nu}\sigma_{i^{\prime}_{\nu}}\sigma_{j^{\prime}_{\nu}}\bigg)
=\alpha^{\prime}\sum_{n=1}^{\infty}\frac{1}{2n}
\tanh^{2n}(\beta^{\prime})(1-\langle q^{2}_{2n}\rangle)\ .
\end{equation}
in integral average with respect to the degree of connectivity.
The random 
variables $P_{\alpha^{\prime}}, \{J^{\prime}_{\nu}\}$,
$\{i^{\prime}_{\nu}\},\{j^{\prime}_{\nu}\}$ are independent
copies of the analogous random variables 
in the Hamiltonian contained in $\Omega$.
\end{lemma}
Notice that, in distribution 
\begin{equation}
\label{continuo1}
\beta\sum_{\nu=1}^{P_{\alpha N}}
J_{\nu}\sigma_{i_{\nu}}\sigma_{j_{\nu}}+\beta^{\prime}
\sum_{\nu=1}^{P_{\alpha^{\prime}}}
J^{\prime}_{\nu}\sigma_{i^{\prime}_{\nu}}\sigma_{j^{\prime}_{\nu}}
\sim\beta\sum_{\nu=1}^{P_{\alpha N+\alpha^{\prime}}}
J^{\prime\prime}_{\nu}\sigma_{i_{\nu}}\sigma_{j_{\nu}}
\end{equation}
where $\{J^{\prime\prime}_{\nu}\}$ are independent copies 
of $J$ with probability $\alpha N/(\alpha N+\alpha^{\prime})$ and 
independent copies of $J\beta^{\prime}/\beta$ with probability
$\alpha^{\prime}/(\alpha N+\alpha^{\prime})$. In the right hand
side above, the quenched random variables will be collectively denoted
by $\mathcal{J}^{\prime\prime}$.
Notice also that 
the sum of Poisson random variables 
is a Poisson random variable with
mean equal to the sum of the means, and hence we can write
\begin{equation}
  \label{continuo2}
  A_{t}\equiv\mathbb{E}\ln\Omega\exp\bigg(\beta^{\prime}
  \sum_{\nu=1}^{P_{\alpha^{\prime}t}}
  J^{\prime}_{\nu}\sigma_{i^{\prime}_{\nu}}\sigma_{j^{\prime}_{\nu}}\bigg)
  =\mathbb{E}\ln\frac{Z_{N}(\alpha_{t};
    \mathcal{J}^{\prime\prime})}{Z_{N}(\alpha;\mathcal{J})}\ ,
\end{equation}
where we defined, for $t\in[0, 1]$,
\begin{equation}
  \label{continuo3}
  \alpha_{t}=\alpha+\alpha^{\prime}\frac{t}{N}
\end{equation}
so that $\alpha_{t}\rightarrow\alpha\ \forall \ t$ as $N\to\infty$.\\
\textbf{Proof}.
Let us compute the $t$-derivative of $A_{t}$, as defined in 
(\ref{continuo2})
\begin{equation}
  \frac{d}{dt}A_{t}=
  \mathbb{E}\sum_{m=1}^{\infty}\frac{d}{dt}\pi_{\alpha^{\prime} t}(m)
  \ln\sum_{\sigma}\exp\bigg(\beta^{\prime}\sum_{\nu=1}^{m}
  J^{\prime}_{\nu}\sigma_{i^{\prime}_{\nu}}
  \sigma_{j^{\prime}_{\nu}}\bigg)\ .
\end{equation}
Using the following
elementary property of the Poisson measure
\begin{equation}\label{poisson}
  \frac{d}{dt}\pi_{t\zeta}(m)=\zeta(\pi_{t\zeta}(m-1)-\pi_{t\zeta}(m))
\end{equation}
we get
\begin{eqnarray*}
  \frac{d}{dt}A_{t}&=&
  \alpha^{\prime}\mathbb{E}\sum_{m=0}^{\infty}
  [\pi_{\alpha^{\prime} t}(m-1)
  -\pi_{\alpha^{\prime} t}(m)]
  \ln\sum_{\sigma}\exp(\beta^{\prime}\sum_{\nu=1}^{m}J^{\prime}_{\nu}
  \sigma_{i^{\prime}_{\nu}}\sigma_{j^{\prime}_{\nu}})\\
  {}&=&\alpha^{\prime}\mathbb{E}\ln\sum_{\sigma}
  \exp(\beta^{\prime} J^{\prime}\sigma_{i^{\prime}_{m}}
  \sigma_{j^{\prime}_{m}})
  \exp(\beta^{\prime}\sum_{\nu=1}^{P_{\alpha^{\prime} t}}J^{\prime}_{\nu}
  \sigma_{i^{\prime}_{\nu}}\sigma_{j^{\prime}_{\nu}})\\
  {}&{}&\hspace{3cm}-\alpha^{\prime}\mathbb{E}\ln\sum_{\sigma}
  \exp(\beta^{\prime}\sum_{\nu=1}^{P_{\alpha^{\prime} t}}J^{\prime}_{\nu}
  \sigma_{i^{\prime}_{\nu}}\sigma_{j^{\prime}_{\nu}})\\
  {}&=&\alpha^{\prime}\mathbb{E}\ln\Omega_{t}
  \exp(\beta^{\prime} J^{\prime}\sigma_{i^{\prime}_{m}}
  \sigma_{j^{\prime}_{m}})\ ,
\end{eqnarray*}
where the average $\Omega_{t}$ is associated with the 
Hamiltonian plus the $t$-dependent weights in the exponential
in (\ref{continuo2}).
Now use the following identity
$$
\exp(\beta^{\prime} J^{\prime}\sigma_{i}\sigma_{j})
=\cosh(\beta^{\prime} J^{\prime})+\sigma_{i}\sigma_{j}
\sinh(\beta^{\prime} J^{\prime})
$$
to get
\begin{equation}
  \frac{d}{dt}A_{t}
  =\alpha^{\prime}\mathbb{E}\ln\Omega_{t}
  [\cosh(\beta^{\prime} J^{\prime})(1+\tanh(\beta^{\prime} 
  J^{\prime})\sigma_{i^{\prime}_{m}}
  \sigma_{j^{\prime}_{m}})]\ .
\end{equation}
We now expand the logarithm in power series and see that,
when $N\to\infty$, as $\alpha_{t}\to\alpha$
the result does not depend on $t$,
wherever the expectation $\Omega_{t}$
is continuous
as a function of the parameter $t$. 
From the comments that preceded
the current proof, formalized 
in (\ref{continuo1})-(\ref{continuo2})-(\ref{continuo3}), 
this is the same as assuming that
$\Omega$ is regular as a function of $\alpha$,
because
$J^{\prime\prime}\to J$ in the sense that in the large
$N$ limit $J^{\prime\prime}$ can only take
the usual values $\pm 1$ since the probability of being
$\pm \beta^{\prime}/\beta$ becomes zero. 
Therefore integrating
against $t$ from 0 to 1 is the same as multiplying by 1.
Due to the symmetric distribution of $J$,
the expansion of the logarithm yields the right hand side of
(\ref{stability}),
where the odd powers are missing. $\Box$

Notice that the left hand side of (\ref{stability}) 
can be written, according to our notations,
 $\mathbb{E}\ln\Omega\exp
 (-\beta^{\prime}\hat{H}_\sigma(\alpha^{\prime},\mathcal{J}^{\prime}))$. %,
%dropping the index $\sigma$  in the  definition of $\hat{H}$.
We want now to consider the statement of Lemma \ref{lemma}
in the case of two independent perturbations, assuming
we are always in the thermodynamic limit.
The consequent generalization obtained by using twice the
fundamental theorem of calculus simply gives (in integral average)
\begin{equation}\label{twopert}
\mathbb{E}\ln\Omega[\exp(-\beta^{\prime}_1\hat{H}_\sigma
(\alpha^{\prime}_{1};\mathcal{J}_{1}^{\prime})
-\beta^{\prime}_2\hat{H}_\sigma(\alpha^{\prime}_{2};
\mathcal{J}_{2}^{\prime}))]=(\alpha^{\prime}_{1}
+\alpha^{\prime}_{2})A\ ,
\end{equation}
where $A$ again does not depend,
in the thermodynamic limit, on 
$\alpha_1^{\prime},\alpha_2^{\prime}$, and has 
the same form as the right hand side of
(\ref{stability}) although the explicit for of $A$ is
not important here. In the equation above,
assumed to be taken in the thermodynamic limit, $\Omega$ 
is the Gibbs measure associated with the unperturbed
Hamiltonian of the original model, and the same holds
for the averages appearing in $A$, just like in the previous lemma.
Clearly at this point we have 
$$
\frac{\partial^{2}}{\partial\alpha^{\prime}_{1}
  \partial\alpha^{\prime}_{2}}
\mathbb{E}\ln\Omega[\exp(-\beta^{\prime}_{1}
\hat{H}_\sigma(\alpha^{\prime}_{1};\mathcal{J}_{1}^{\prime})
-\beta^{\prime}_{2}\hat{H}_\sigma(\alpha^{\prime}_{2};\mathcal{J}_{2}^{\prime}))]=0\ ,
$$
whenever the derivatives exist, i.e. with the possible
exception of the zero measure set, by convexity.
A simple computation yields
\begin{multline*}
  \frac{\partial^{2}}{\partial\alpha^{\prime}_{1}
    \partial\alpha^{\prime}_{2}}
  \mathbb{E}\ln\Omega[\exp(-\beta^{\prime}_1\hat{H}_\sigma(\alpha^{\prime}_{1};
  \mathcal{J}_{1}^{\prime})
  -\beta^{\prime}_2\hat{H}_\sigma(\alpha^{\prime}_{2};\mathcal{J}_{2}^{\prime}))]=0\\
  =\mathbb{E}\ln\Omega^{\prime}[\exp(\beta^{\prime}_{1}J^{\prime}_{1}
  \sigma_{i_{1}}\sigma_{j_{1}}+
  \beta^{\prime}_{2}J^{\prime}_{2}\sigma_{i_{2}}\sigma_{j_{2}}]\\
  -\mathbb{E}\ln\Omega^{\prime}[\exp(\beta^{\prime}_{1}J^{\prime}_{1}
  \sigma_{i_{1}}\sigma_{j_{1}})]\Omega^{\prime}[\exp(\beta^{\prime}_{2}
  J^{\prime}_{2}
  \sigma_{i_{2}}\sigma_{j_{2}})]
\end{multline*}
Every time a derivative with respect to a pertubing parameter is taken, 
the relative perturbation is added to the weights of the measure
$\Omega$, which hence replaced by a perturbed measure
denoted by $\Omega^{\prime}$.
If the pertubation is small (like in our case, as
explained in the previous lemma) it disappears from the measure
in the thermodynamic limit. Recall that this holds with the usual limitation, i.e.
only in integral average, because each derivative exists only almost everywhere,
and meaningful equalities are hence proven only under integration over any given
interval.
Hence both in the equation above and
in the next calculation $\beta_1^{\prime},\beta_2^{\prime}$ are
not in the measure $\Omega^{\prime}$, and we get
\begin{multline}\label{recap}
  \frac{\partial^{2}}{\partial(\beta^{\prime}_{1}J_{1})
    \partial(\beta^{\prime}_{2}J_{2})}
  \mathbb{E}\ln\Omega^{\prime}[\exp(\beta^{\prime}_{1}J^{\prime}_{1}
  \sigma_{i_{1}}\sigma_{j_{1}}+
  \beta^{\prime}_{2}J^{\prime}_{2}\sigma_{i_{2}}\sigma_{j_{2}}]\\
  =\mathbb{E}\Omega^{\prime\prime}(\sigma_{i_{1}}\sigma_{j_{1}})
  -\mathbb{E}\Omega^{\prime\prime}
  (\sigma_{i_{1}})\Omega^{\prime\prime}(\sigma_{j_{1}})=0\ ,
\end{multline}
again in integral average with respect to all the parameters
$\alpha_{1}^{\prime},\alpha_{2}^{\prime},\beta_{1}^{\prime},\beta_{2}^{\prime}$.
These new derivatives with respect to $\beta_{1}^{\prime},\beta_{2}^{\prime}$
again introduce further perturbations in the weights, 
which is why we used the notation $\Omega^{\prime\prime}$,
but as usual in the thermodynamic limit they vanish (in integral average).
The first line of (\ref{recap}) gives us the 
generator of a family of relations that
we will obtain by means of an expansion in powers
of $\beta_1^{\prime},\beta_2^{\prime}$.
The second line of the equation formulates the self-averaging
(with respect to the Gibbs measure)
implied by stochastic stability.

So we proceed starting from the next lemma and the next theorem,
summarizing what we just discussed. We want here to remind ourselves of the
presence of the perturbations, vanishing only in the thermodynamic
limit and only with probability one in the space of all parameters, 
by denoting any perturbed
Gibbs expectation with $\Omega^{\prime}$, independently
of the perturbations.
\begin{lemma}\label{prima} Let $\Omega^{\prime}$
be the Gibbs measure including two independent perturbations
of the form 
$$
\hat{H}_\sigma(\alpha^{\prime};\mathcal{J}^{\prime})=\sum_{\nu=1}^{P_{\alpha^{\prime}}}
J^{\prime}_{\nu}\sigma_{i_{\nu}}\sigma_{j_{\nu}}
$$
with parameters $\alpha_1^{\prime},\alpha_2^{\prime}, 
\beta_1^{\prime},\beta_2^{\prime}$ like in (\ref{twopert}).
Then, recalling that $m$ is the magnetization, 
the following self-averaging (with respect to the Gibbs measure)
identity
\begin{equation}\label{phi}
  \mathbb{E}\{\Omega^{\prime}
  (m^{2})
  -[\Omega^{\prime}(m)]^{2}\}=0 
\end{equation}
holds in the thermodynamic limit in integral average with respect
to the perturbing parameters
$\alpha_1^{\prime},\alpha_2^{\prime},\beta_1^{\prime},\beta_2^{\prime}$.
\end{lemma}
We will see again that in the first line of equation (\ref{recap})
the expression remains zero even without the derivative.
In fact the generator of the identities we want to prove
is expressed in the following 
\begin{theorem}\label{thm-generatore} 
In the thermodynamic limit the following identity holds
\begin{eqnarray}\label{eq-generatore}
&& \mathbb{E}\ln 
\Omega^{\prime}(\exp(\beta_1^{\prime} 
J^{\prime}_1\sigma_{i_1}\sigma_{j_1}
+\beta_2^{\prime} J^{\prime}_2\sigma_{i_2}\sigma_{j_2}))= \\
\nonumber
&&\mathbb{E}\ln 
\Omega^{\prime}(\exp(\beta_1^{\prime}
J_1^{\prime}\sigma_{i_1}\sigma_{j_1}))
+\mathbb{E}\ln 
\Omega^{\prime}(\exp(\beta_2^{\prime}
J_2^{\prime}\sigma_{i_2}\sigma_{j_2}))\ .
\end{eqnarray}
in integral average with respect to 
$\alpha_1^{\prime},\alpha_2^{\prime}, \beta_1^{\prime},\beta_2^{\prime}$.
\end{theorem}
The relations we will derive are a simple consequence
of this theorem, and fomalized in the next
\begin{corollary} In the thermodynamic limit, 
in integral average with respect to the perturbing parameters 
$\alpha_1^{\prime},\alpha_2^{\prime},\beta_1^{\prime},\beta_2^{\prime}$, we have
$$
\sum_{a=0}^{\min\{r,s\}}(-)^{a+1}\frac{(2r+2s-a-1)!}{a!(2r-a)!(2s-a)!}
\langle q^2_{2r}q^2_{2s}\rangle^{\prime}_a=0  
\ \ \forall\ r,s \in \mathbb{N}\ ,
$$
where the subscript a in the global average
$\langle\cdot\rangle^{\prime}_a=\mathbb{E}\Omega^{\prime}_{a}$ 
means that precisely $a$ replicas are in common
among those in $q_{2r}$ and those in $q_{2s}$.
\end{corollary}
{\bf Proof}. The following shorthand will be employed
$$
t_1=\tanh(\beta^{\prime}_1 J^{\prime}_1)\ ,\ 
t_2=\tanh(\beta^{\prime}_2 J^{\prime}_2)\ ,
$$
$$
\ \Omega_1=\Omega^{\prime}(\sigma_{i_1}
\sigma_{j_1})\ ,\
\Omega_2=\Omega^{\prime}(\sigma_{i_2}\sigma_{j_2})\ , \ 
\Omega_{12}=\Omega^{\prime}(\sigma_{i_1}
\sigma_{j_1}\sigma_{i_2}\sigma_{j_2})
$$ 
and
$$
W=\Omega^{\prime}(\exp
(\beta_1^{\prime} J^{\prime}_1\sigma_{i_1}\sigma_{j_1}
+\beta_2^{\prime} J^{\prime}_2\sigma_{i_2}\sigma_{j_2}))\ .
$$
Observe that, if we let $\delta=1,2$,
\begin{equation}
\label{derivative}
\frac{\partial}{\partial {\beta J^{\prime}_\delta}}
=(1-t_{\delta}^2)\frac{\partial}{\partial {t_{\delta}}}\ .
\end{equation}
Now,
$$
\ln W =
\ln(1+t_1\Omega_1+t_2\Omega_{2}+t_1t_2\Omega_{12})+
\ln\cosh\beta J^{\prime}_1+\ln\cosh\beta J^{\prime}_2
$$
and
\begin{multline*}
\ln(1+t_1\Omega_1+t_2\Omega_{2}+t_1t_2\Omega_{12})=\\
\sum_{n=1}^{\infty}\sum_{l=0}^{n}\sum_{m=0}^{l}\frac{(-)^{n+1}}{n}
\binom{n}{l}\binom{l}{m}
t_1^{n-l+m}t_2^{n-m}\Omega_{1}^{m}\Omega_2^{l-m}\Omega_{12}^{n-l}\\
=\sum_{n,l,m}(-)^{n+1}\frac{(n-1)!}{(n-l)!(l-m)!m!}
t_1^{n-l+m}t_2^{n-m}\Omega_{1}^{m}\Omega_2^{l-m}\Omega_{12}^{n-l}\ .
\end{multline*}
The derivatives in (\ref{recap}) kill the two terms with the
hyperbolic cosines, and from (\ref{derivative}) we know that we can
replace the derivatives with respect to $\beta J^{\prime}_\delta$ with
the derivatives with respect to $t_\delta$, $\delta=1,2$.  Notice that
the logarithm just expanded is zero for $t_{1}=0$ and for $t_{2}=0$,
therefore as its derivative like in (\ref{recap}) is zero, the
logarithm itself is zero. This is why Theorem \ref{thm-generatore}
holds, being (\ref{eq-generatore}) just the integral of the second
line in (\ref{recap}).

Thanks to (\ref{q}), if we put
$$
n-l+m=r\ ,\ n-m = s\ ,\ n-l = a
$$
we get
$$
\sum_{r,s}\mathbb{E}[t_1^{r}t_2^{s}]\sum_{a=0}^{\min\{r,s\}}(-)^{a+1}
\frac{(r+s-a-1)!}{a!(r-a)!(s-a)!}
\langle q^{2}_{r}q^{2}_{s}\rangle^{\prime}_a=0 
$$
where, recall, $\langle\cdot\rangle_a$ means that $a$ replicas are in common
among those in $q_{r}$ and those in $q_{s}$. 
Hence the statement of the theorem to be proven
$$
\sum_{a=0}^{\min\{2r,2s\}}(-)^{a+1}\frac{(2r+2s-a-1)!}{a!(2r-a)!(2s-a)!}
\langle q^{2}_{2r}q^{2}_{2s}\rangle^{\prime}_a=0\ ,
$$
where only the terms with an even number of replicas in each overlap
survive because of the symmetry of the variables $J^{\prime}_{1},J^{\prime}_{2}$
in $t_{1},t_{2}$. $\Box$



%%%%%%%%%%%%%%%%%%%%%%%%%%%%%%%%%%%%%%%
%%%%%%%%%%%%%%%%%%%%%%%%%%%%%%%%%%%%%%%
\subsection{Generalization to smooth functions of multi-overlaps}


The fact  that in our formulas we  always got the square  power of the
overlaps  is  due  to  the   fact  that  the  Hamiltonian  has  2-spin
interactions.  Everything  we did so  far could then be  reproduced in
the  case  of $p$-spin  interactions,  and  we  would obtain  the  same
relations just derived, except the  overlaps would appear with the power
$p$ instead of  2.  Clearly the perturbation needed in  this case is a
$p$-spin  perturbation too.   More  in general,  we  could consider  a
Hamiltonian consisting of the  sum (over $p$) of $p$-spin Hamiltonians
for  any integer  $p$.  Then  we could  perturb each  of  the $p$-spin
Hamiltonians with its proper  small $p$-spin perturbation, and add all
these perturbations to  the system. Clearly we have  to make sure that
all the terms in this whole Hamiltonian are weighted with sufficiently
small weights so to  have the necessary convergence.  More explicitly,
the   perturbed   Hamiltonian   is   
$$
H_{N}(\sigma,\alpha;\mathcal{J})=
-\sum_{p}\bigg[a_{p}\sum_{\nu=1}^{P^{(p)}_{\alpha
    N}}J_{\nu}\sigma_{i^{1}_{\nu}}\cdots\sigma_{i^{p}_{\nu}}+b_{p}
\lambda_{p}\sum_{\nu=1}^{P^{\prime
    (p)}_{\alpha^{\prime}}}J^{\prime}_{\nu}\sigma_{j^{1}_{\nu}}
\cdots\sigma_{j^{p}_{\nu}}\bigg]\ ,          
$$
where
$\sum_{p}|a_{p}|^{2}=\sum_{p}|b_{p}|^{2}=1$, the  notation for all the
quenched  variables is the  usual one,  and $\{\lambda_{p}\}$  are the
independent perturbing real parameters.

It is not surprising then that we can state
\begin{corollary}\label{corollary}
The following constraints hold in integral average with respect to
the set of all the perturbing parameters
$$
\sum_{a=0}^{\min\{2r,2s\}}(-)^{a+1}\frac{(2r+2s-a-1)!}{a!(2r-a)!(2s-a)!}
\langle q^m_{2r}q^n_{2s}\rangle^{\prime}_a=0 
\ \ \ \forall\ r,s,m,n \in \mathbb{N}\ 
$$
in the thermodynamic limit.
\end{corollary}
Again, this corollary can be seen as a consequence of 
a self-averaging property, namely
\begin{equation}
\mathbb{E}\Omega^{\prime}(\sigma_{i^1_1}\cdots\sigma_{i^m_1}
\sigma_{j^1_1}\cdots\sigma_{j^n_1})-
\mathbb{E}[\Omega^{\prime}(\sigma_{i^1_1}\cdots\sigma_{i^m_1})
\Omega^{\prime}(\sigma_{j^1_1}\cdots\sigma_{j^n_1})]=0\ .
\end{equation}

Therefore we can replace each overlap by any smooth function
of the relative replicas in the statement of the corollaries.

As a last remark, notice that in \cite{flt} the strategy consisted
in using the fact the second derivative of the free energy 
with respect to the ``perturbing inverse temperatures'' 
($\beta^{\prime}_{1},\beta^{\prime}_{2}$ in our case) is bounded
to deduce self-averaging, and from the latter the identities.
Here we used the connectivity as opposed to the inverse temperature
to analyze stochastic stability, we then showed that the latter implies
the self-averaging of \cite{flt}.
So we obtained a comparison between self-averaging and stochastic
stability (of a quite general validity), both providing a 
precious factorization of the Gibbs measure, and we also
obtained that if any of the two is given, we know how to 
derive the same constraints.



%%%%%%%%%%%%%%%%%%%%%%%%%%%%%%%%%%%%%%%%
%%%%%%%%%%%%%%%%%%%%%%%%%%%%%%%%%%%%%%%%

\section{Self-averaging of the quenched-Gibbs measure}
\label{energysection}

Roughly speaking, if a convex random function does not
fluctuate much, then its derivative does not fluctuate much
either, with the exception of bad cases. This is well
explained in Proposition 4.3 of \cite{t5} and
Lemma 8.10 of \cite{bov2}. 
We are not interested
in general theorems here, in our case the convex function
is the free energy density, and 
we only need to know that it is self-averaging
(in the sense that the random free energy density
does not fluctuate around its quenched expectation,
in the thermodynamic limit). In the case
of finite connectivity random spin systems like the VB model, 
a detailed proof
of this can be found in \cite{gt1}. The derivative
of the free energy density (times $-\beta)$ with respect to $-\beta$
is, in full generality, the expectation of the 
internal energy density $u_{N}=H_{N}/N$.
Like in \cite{guerra2} and in 
section 2 of \cite{gg}, 
we have therefore this further self-averaging (in integral average with
respect to $\beta$)
$$
\lim_{N\to\infty}[\langle u_{N}^{2}\rangle
-\langle u_{N}\rangle^{2}]=0
$$
which implies (due to Schwartz inequality)
\begin{equation}
\label{auto}
\lim_{N\to\infty}\langle u^{(1)}_{N}\phi_{s}\rangle
=\lim_{N\to\infty}\langle u_{N}\rangle\langle\phi_{s}\rangle
\end{equation}
for any bounded function  $\phi_{s}$ of $s$ replicas, and
$u^{(1)}_{N}$ is the internal energy density in the configuration space
of the replica 1. More precisely,
the spin-configuration space is $\{-1,1\}^{N}= \Sigma$, and we
consider a bounded function $\phi_{s}$ of $s$ replicas, 
i.e. $\phi_{s}:\Sigma^{s}\to \mathbb{R}$.
The product space $\Sigma^{s}$ (``the space of the replicas'')
is equipped with the product Gibbs measure (``replica measure'')
$\Omega$, but the quenched variables 
are the same in each factor of the product space, and this means
that the measure $\langle\cdot\rangle=\mathbb{E}\Omega(\cdot)$
on the product space $\Sigma^{s}$ is not a product measure.
So $f^{(1)}_{N}$ is the free energy in the space which
is the first factor in the product space $\Sigma^{s}$.
Notice that $\Sigma$ has the cardinality of the continuum 
in the thermodynamic limit $N\to\infty$.
%Let us recall that apices denumerate replicas for the spins and the
%Hamiltonian, while they are just regular exponents in the case of 
%overlaps, where the replicas are counted or listed in the sub-index. 

Now 
we want to perturb the Hamiltonian 
$$
-\beta H_{N}(\sigma)\ \longrightarrow\ 
-\beta H_{N}(\sigma)+\beta^{\prime}\sum_{\nu=1}^{P^{\prime}\alpha}
J^{\prime}_{\nu}\sigma_{i^{\prime}_{\nu}}\sigma_{j^{\prime}_{\nu}} \ ,
$$
and consider
the derivative with respect to the perturbing parameter,
as we did in the previous section
in order to obtain an expansion in powers of $\beta^{\prime}$
with coefficients not depending on  $\beta^{\prime}$
in the thermodynamic limit. As before, recall that we always need to 
take derivatives (with respect to the inverse temperature this time),
which by convexity exist only almost everywhere, and therefore 
we will obtain equality only integrating back over a given interval \cite{talabook}.

We are going to prove, first of all, the following
\begin{theorem}\label{quattro} 
For a given bounded function $\phi_{s}$ of $s$ replicas,
the following relation, holding in integral average with respect to the 
inverse temperature $\beta$, constrains the distribution
of the 4-overlap
  \begin{multline*}
    \frac{s(s+1)(s+2)}{3!}
    \langle q^{2}_{1,s+1,s+2,s+3}\phi_{s}\rangle
    -\frac{s(s+1)}{2!}
    \sum_{a}^{2,s}\langle q^{2}_{1,a,s+1,s+2}\phi_{s}\rangle\\
    +s\sum_{a<b}^{2,s}\langle q^{2}_{1,a,b,s+1}\phi_{s}\rangle
    -\sum_{a<b<c}^{2,s}
    \langle q^{2}_{1,a,b,c}\phi_{s}\rangle
    =\langle q^{2}_{1234}\rangle\langle \phi_{s} \rangle \ .
  \end{multline*}
\end{theorem}
The proof is straightforward but long, and it will 
be splitted into several steps.

Let us consider the right hand side of (\ref{auto}). 
Put $t=\tanh(\beta^{\prime})$, $q_0=1$,
and let us just indicate the number
of replicas in the overlaps, rather than denumerating them all.
The presence of the perturbation implies that the pressure and
the free energy are functions, $\tilde{p}_{N}(\beta,\beta^{\prime},\alpha)$
and $\tilde{f}_{N}(\beta,\beta^{\prime},\alpha)$ respectively,
of both $\beta$ and $\beta^{\prime}$,
and recall also that according to our notations
$\tilde{p}_{N}(\beta,\beta^{\prime},\alpha)=
-\beta \tilde{f}_{N}(\beta,\beta^{\prime},\alpha)$.
Let us prove the next
\begin{lemma}\label{energy} The derivative
of the (perturbed) pressure $\tilde{p}_N(\beta,\beta^{\prime},\alpha)$
with respect to the perturbing parameter $\beta^{\prime}$
has the following form as a series in powers of
$t=\tanh(\beta^{\prime})$
  \begin{equation}
    \partial_{\beta^{\prime}}\tilde{p}_{N}(\beta,\beta^{\prime},\alpha)
    =-\alpha\sum_{n=0}^{\infty}
    t^{2n+1}(\langle q^{2}_{2n}\rangle-\langle q^{2}_{2n+2}\rangle)\ .
  \end{equation}
\end{lemma}
{\bf Proof.} We have
\begin{eqnarray*}
  \partial_{\beta^{\prime}}\tilde{p}_{N}(\beta,\beta^{\prime},\alpha)
  &=&-\sum_{m=1}^\infty
  \pi_{\alpha}(m)\sum_{\nu=1}^m\langle 
  J^{\prime}_\nu\sigma_{i^{\prime}_\nu}
  \sigma_{j^{\prime}_{\nu}}\rangle_m\\
  &=&-\sum_{m=1}^\infty
  m\pi_{\alpha}(m)\langle J^{\prime}_m\sigma_{i^{\prime}_m}
  \sigma_{j^{\prime}_{m}}\rangle_m\\
  &=& -\alpha\sum_{m=1}^\infty
  \pi_{\alpha}(m-1)\langle J^{\prime}_m\sigma_{i^{\prime}_m}
  \sigma_{j^{\prime}_{m}}\rangle_m
\end{eqnarray*}
where the sub $m$ indicates that the variable $P^{\prime}_{\alpha}$
has been fixed to $m$.
It is easy to see that
\begin{equation}\label{acca1}
  \langle J^{\prime}_m\sigma_{i^{\prime}_m}
  \sigma_{j^{\prime}_{m}}\rangle_m=
  \mathbb{E}\frac{\Omega(J^{\prime}_m\sigma_{i^{\prime}_m}
  \sigma_{j^{\prime}_{m}}
    \exp(\beta J^{\prime}_m\sigma_{i^{\prime}_m}
    \sigma_{j^{\prime}_{m}}))_{m-1}}{\Omega(
    \exp(\beta J^{\prime}_m\sigma_{i^{\prime}_m}
    \sigma_{j^{\prime}_{m}}))_{m-1}}\ .
\end{equation}
Hence
\begin{equation}\label{acca2}
  \partial_{\beta^{\prime}}\tilde{p}_{N}(\beta,\beta^{\prime},\alpha)
  =-\alpha\mathbb{E}J^{\prime}
  \frac{t+w}{1+tw}\ ,\ w\equiv
  \Omega(\sigma_{i^{\prime}_m}\sigma_{j^{\prime}_{m}})\ ,
\end{equation}
according to the usual notations. Now a simple expansion
(that we will explicitly write in the next lemma)
of $(1+tw)^{-1}$ in powers of $t$ yields
\begin{equation}\label{phioutside}
 \partial_{\beta^{\prime}}\tilde{p}_{N}(\beta,\beta^{\prime},\alpha)
  =-\alpha\sum_{n=0}^{\infty}
  t^{2n+1}(\langle q^{2}_{2n}\rangle-\langle q^{2}_{2n+2}\rangle)\ .
\end{equation}
So the lemma is proven and we have an expression for
the right hand side of (\ref{auto}), if we just
multiply the average of the multi-overlaps by
the average of $\phi_s$. $\Box$

Let us now consider the left hand side of (\ref{auto}),
recalling that $\phi_{s}$ is a function of $s$ replicas,
that indices in the spins indicate which factor of the
product space $\Sigma^{s}$ (which replica) the spin belongs to, 
and that the energy density is assumed
to be taken in the first replica.
\begin{lemma}\label{espansione}
Recalling that $w\equiv\Omega(\sigma_{i^{\prime}_m}
\sigma_{j^{\prime}_m})$, we have
  \begin{multline*}
     \langle u^{(1)}_{N} \phi_{s} \rangle=
    -\alpha t\mathbb{E}\{\Omega[\phi_{s}(1+J^{\prime}t^{-1}\sigma^{(1)}_{i_1}
    \sigma^{(1)}_{j_1})\times \\
    (1+J^{\prime}\sum_a^{2,s}\sigma^{(a)}_{i_1}\sigma^{(a)}_{j_1} t+
    \sum_{a<b}^{2,s}\sigma_{i_1}^{(a)}\sigma_{i_1}^{(b)}
    \sigma_{j_1}^{(a)}\sigma_{j_1}^{(b)}t^2+\\
    J^{\prime}
    \sum_{a<b<c}^{2,s}\sigma_{i_1}^{(a)}
    \sigma_{i_1}^{(b)}\sigma_{i_1}^{(c)}\sigma_{j_1}^{(a)}
    \sigma_{j_1}^{(b)}\sigma_{j_1}^{(c)}t^3
    +\cdots)]\times \\
    (1-J^{\prime}stw+\frac{s(s+1)}{2!}t^{2}w^{2}-
    J^{\prime}\frac{s(s+1)(s+2)}{3!}t^3w^3\\
    +\frac{s(s+1)(s+2)(s+3)}{4!}t^4w^4- \cdots)\}\ .
  \end{multline*}
\end{lemma} 
{\bf Proof}. From the proof of the previous lemma, 
in particular equations (\ref{acca1})-(\ref{acca2}), and
by definition of replica measure, we immediately get
\begin{equation}\label{phiinside}
  \langle u^{(1)}\phi_{s} \rangle=
  -\alpha\mathbb{E}
  \frac{\Omega[ J^{\prime}\sigma^{(1)}_{i_1}\sigma^{(1)}_{j_1}
    \exp(\beta J^{\prime}(\sigma^{(1)}_{i_1}\sigma^{(1)}_{j_1}
    +\cdots+\sigma^{(s)}_{i_1}\sigma^{(s)}_{j_1}))\phi_{s}]}
  {\Omega^s(\exp (\beta J^{\prime}\sigma_{i_1}\sigma_{j_1}))}\ ,
\end{equation}
that we rewrite as
\begin{equation}
  \langle u^{(1)}\phi_{s} \rangle=
  -\alpha\mathbb{E}t
  \frac{\Omega[(1+J^{\prime}t^{-1}\sigma^{(1)}_{i_1}\sigma^{(1)}_{j_1})\prod_{a=2}^s
    (1+J^{\prime}t\sigma^{(a)}_{i_1}\sigma^{(a)}_{j_1})\phi_{s}]}
  {(1+J^{\prime}tw)^s}\ .
\end{equation}
Let us write explicitly the power expansion of the denominator,
that we omitted in the previous lemma
\begin{multline*}
  \frac{1}{(1+J^{\prime}tw)^s}=1-J^{\prime}stw+
  \frac{s(s+1)}{2!}t^{2}w^{2}-\\
  J^{\prime}\frac{s(s+1)(s+2)}{3!}t^3w^3
  +\frac{s(s+1)(s+2)(s+3)}{4!}t^4w^4\cdots\ .
\end{multline*}
It is also clear that
\begin{multline*}
  \prod_{a=2}^s(1+J^{\prime}t\sigma^{(a)}_{i_1}\sigma^{(a)}_{j_1})
  =1+J^{\prime}\sum_a^{2,s}\sigma^{(a)}_{i_1}\sigma^{(a)}_{j_1} t+
  \sum_{a<b}^{2,s}\sigma_{i_1}^{(a)}\sigma_{i_1}^{(b)}
  \sigma^{(a)}_{j_1}\sigma^{(b)}_{j_1}t^2\\
  +J^{\prime}\sum_{a<b<c}^{2,s}\sigma_{i_1}^{(a)}
  \sigma_{i_1}^{(b)}\sigma_{i_1}^{(c)}\sigma^{(a)}_{j_1}
  \sigma^{(b)}_{j_1}\sigma^{(c)}_{j_1}t^3
  +\cdots\ .
\end{multline*}
Gathering all the ingredients completes the proof of the lemma. $\Box$

We are now able to compare the two sides of (\ref{auto}),
and see what the self-averaging of the internal energy density
in the thermodynamic limit brings.

Equating the expressions computed in the 
last two lemmas gives
\begin{multline}\label{sviluppo}
  \sum_{n=0}^{\infty}
  t^{2n}(\langle q^{2}_{2n}\rangle
  -\langle q^{2}_{2n+2}\rangle)\langle \phi_{s} \rangle
  =\mathbb{E}\{\Omega[\phi_{s}(1+Jt^{-1}\sigma^{(1)}_{i_1}
  \sigma^{(1)}_{j_1})\\
  (1+J^{\prime}\sum_{a}^{2,s}\sigma^{(a)}_{i_1}\sigma^{(a)}_{j_1} t+
  \sum_{a<b}^{2,s}\sigma_{i_1}^{(a)}\sigma_{i_1}^{(b)}
  \sigma_{j_1}^{(a)}\sigma_{j_1}^{(b)}t^2+\\
  J^{\prime}
  \sum_{a<b<c}^{2,s}\sigma_{i_1}^{(a)}
  \sigma_{i_1}^{(b)}\sigma_{i_1}^{(c)}\sigma_{i_1}^{(a)}\sigma_{i_1}^{(b)}
  \sigma^{(c)}_{j_{1}}t^3+\\
  \cdots+J^{\prime s-1}t^{s-1}\sigma_{i_1}^{(2)}\cdots\sigma_{i_1}^{(s)}
  \sigma_{j_1}^{(2)}\cdots\sigma_{j_1}^{(s)})]\\
  (1-J^{\prime}stw+\frac{s(s+1)}{2!}t^2w^2-
  J^{\prime}\frac{s(s+1)(s+2)}{3!}t^3w^3\\
  +\frac{s(s+1)(s+2)(s+3)}{4!}t^4w^4-\cdots)\}\ .
\end{multline}
The equality holds for any smooth function $\phi_{s}$
(typical interesting information is obtained 
for $\phi_{s}\equiv1$ or $\phi_{s=2n}=q^{2}_{2n}$), so that we get
equalities between expressions
involving averages of (squared) overlaps. 

Let us see in detail what information we can get from the lowest orders.

Denote by $\mathbb{E}(\cdot|\mathcal{A}_s)$ the conditional
expectation with respect to the sigma-algebra $\mathcal{A}_s$
generated by the overlaps of $s$ replicas. Let us show that the usual
\cite{gg} Ghirlanda-Guerra identities for the overlap hold in our
quite general case too:
\begin{proposition} The Ghirlanda-Guerra relation holds 
  \begin{equation}\label{ghigu}
    \mathbb{E}(q^{2}_{a,s+1}|\mathcal{A}_s)=
    \frac1s\langle q^{2}_{12}\rangle
    +\frac1s\sum_{b\neq a} q^{2}_{a b}
  \end{equation}
in integral average with respect to the inverse temperature $\beta$.
\end{proposition}
{\bf Proof}. In the expansion (\ref{sviluppo}), where only the terms of 
even order survive due to the symmetry of the variables $J$,
at the lowest order in $t$ one gets 
\begin{eqnarray*}
  \langle\phi_{s}\rangle-
  \langle q^{2}_{12}\rangle\langle\phi_{s}\rangle &=&
  \langle\phi_{s}\rangle-s\mathbb{E}
  [\omega(\sigma^{(1)}_{i_1}\sigma^{(1)}_{j_1})w\phi_{s}]
  +\sum_{a}^{2,s}\mathbb{E}[\Omega(
  \sigma^{(1)}_{i_1}\sigma^{(a)}_{i_1}\sigma^{(1)}_{j_1}\sigma^{(a)}_{j_1})\phi_{s}]\\
  {} &=& \langle\phi_{s}\rangle 
  -s\langle q^{2}_{1,s+1}\phi_{s}\rangle
  +\sum_{a}^{2,s}\langle q^{2}_{1a}\phi_{s}\rangle\ ,
\end{eqnarray*}
which is precisely what is stated in (\ref{ghigu}),
(see \cite{talabook}), immediately
completing the proof of the proposition. $\Box$

So the usual Ghirlanda-Guerra
identities for 2-overlaps are recovered and proven to hold in 
dilute spin glasses too.

At the next order we get instead
\begin{multline}\label{ordinedue}
  \langle q^{2}_{12}\rangle\langle\phi_{s}\rangle- \langle
  q^{2}_{1234}\rangle\langle\phi_{s}\rangle=
  \sum_{a<b}^{2,s}\langle q^{2}_{a b}\phi_{s}\rangle+
  \frac{s(s+1)}{2!}\langle q^{2}_{s+1,s+2}\phi_{s}\rangle\\
  -s\sum_a^{2,s}\langle
  q^{2}_{a,s+1}\phi_{s}\rangle-\frac{s(s+1)(s+2)}{3!}
  \langle q^{2}_{1,s+1,s+2,s+3}\phi_{s}\rangle\\
  +\frac{s(s+1)}{2!}\sum_a^{2,s}\langle
  q^{2}_{1,a,s+1,s+2}\phi_{s}\rangle- s\sum_{a<b}^{2,s}\langle
  q^{2}_{1,a,b,s+1}\phi_{s}\rangle\ +\sum_{a<b<c}^{2,s} \langle
  q^{2}_{1,a,b,c}\phi_{s}\rangle\ .
\end{multline}
Now consider the four 2-overlaps terms. 
A simple generalization of the usual Ghirlanda-Guerra relations
\cite{gg}
to the case when two replicas are added to a previously assigned
set of other replicas, tells us that these terms cancel out.
Let us check that explicitly.
\begin{corollary} Relation (\ref{ghigu}) implies
  \begin{equation}\label{ghigu2}
    \mathbb{E}(q^{2}_{s+1,s+2}|\mathcal{A}_{s}) =
    \frac{2}{s+1}\langle q^{2}_{12}\rangle +
    \frac{2}{s(s+1)}\sum_{a<b}^{1,s}q^{2}_{ab}\ ,
  \end{equation}
under the same conditions, i.e. in integral average with respect to
$\beta$.
\end{corollary}
{\bf Proof}.
Let us re-write (\ref{ghigu}) in the case of $s+1$ given
replicas
$$
\mathbb{E}(q^{2}_{s+1,s+2}|\mathcal{A}_{s+1})
=\frac{1}{s+1}\langle q^{2}_{12}\rangle
+\frac{1}{s+1}\sum_{b}^{1,s} q^{2}_{b,s+1}\ .
$$
Now use 
\begin{equation}\label{sigma}
  \mathbb{E}(\mathbb{E}(\cdot|\mathcal{A}_{s+1})|\mathcal{A}_s)
  =\mathbb{E}(\cdot|\mathcal{A}_s)
\end{equation}
to get
\begin{eqnarray*}
  \mathbb{E}(q^{2}_{s+1,s+2}|\mathcal{A}_{s}) & = & 
  \frac{1}{s+1}\langle q^{2}_{12}\rangle
  +\frac{1}{s+1}\sum_{b}^{1,s} \mathbb{E}(q^{2}_{b,s+1}|\mathcal{A}_s)\\
  {} & = & \frac{1}{s+1}\langle q^{2}_{12}\rangle +
  \frac{1}{s+1}\left(\langle q^{2}_{12}\rangle + 
    \frac1s \sum_{b}^{1,s}\sum_{c\neq b}^{1,s}q^{2}_{bc}\right)\ .
\end{eqnarray*}
That is
\begin{equation}
  \mathbb{E}(q^{2}_{s+1,s+2}|\mathcal{A}_{s}) =
  \frac{2}{s+1}\langle q^{2}_{12}\rangle +
  \frac{2}{s(s+1)}\sum_{a<b}^{1,s}q^{2}_{ab}\ ,
\end{equation}
which is what we wanted to prove. $\Box$

Now with (\ref{ghigu}) and (\ref{ghigu2}) in our hands, let us 
take the three 2-overlap terms in the right hand side of (\ref{ordinedue})
\begin{eqnarray*}
  \frac{s(s+1)}{2}\langle q^{2}_{s+1,s+2}\phi_{s}\rangle & = &
  s\langle q^{2}_{12}\rangle\langle\phi_{s}\rangle 
  +\sum_{a<b}^{1,s}\langle q^{2}_{ab}\phi_{s}\rangle \\
  -s\sum_{a}^{2,s}\langle q^{2}_{a,s+1}\phi_{s}\rangle & = & 
  -s\sum_{a}^{1,s}\langle q^{2}_{a,s+1}\phi_{s}\rangle 
  + s\langle q^{2}_{1,s+1}\phi_{s}\rangle \\
  {} & = & -s\langle q^{2}_{12}\rangle\langle\phi_{s}\rangle
  -\sum_{a}^{1,s}\sum_{b\neq a}^{1,s}\langle q^{2}_{ab}\phi_{s}\rangle 
  +\langle q^{2}_{12}\rangle\langle\phi_{s}\rangle 
  +\sum_{a}^{2,s}\langle q^{2}_{1a}\phi_{s}\rangle\\
  \sum_{a<b}^{2,s}\langle q^{2}_{ab}\phi_{s}\rangle & = &
  \sum_{a<b}^{1,s}\langle q^{2}_{ab}\phi_{s}\rangle-
  \sum_{a}^{2,s}\langle q^{2}_{1a}\phi_{s}\rangle\ .
\end{eqnarray*}
The sum of these three terms cleary reduces to 
$\langle q^{2}_{12}\rangle\langle\phi_{s}\rangle$, which is precisely
what we find in the left hand side of (\ref{ordinedue}).
The 2-overlap terms thus cancel out from (\ref{ordinedue}). 
We are hence left with a new relation for 4-overlaps:
\begin{multline*}
  \frac{s(s+1)(s+2)}{3!}
  \langle q^{2}_{1,s+1,s+2,s+3}\phi_{s}\rangle
  -\frac{s(s+1)}{2!}
  \sum_{a}^{2,s}\langle q^{2}_{1,a,s+1,s+2}\phi_{s}\rangle\\
  +s\sum_{a<b}^{2,s}\langle q^{2}_{1,a,b,s+1}\phi_{s}\rangle
  =\langle q^{2}_{1234}\rangle\langle \phi_{s} \rangle
  +\sum_{a<b<c}^{2,s}
  \langle q^{2}_{1,a,b,c}\phi_{s}\rangle\ ,
\end{multline*}
and the proof of Theorem \ref{quattro} is now complete. $\Box$
 

%%%%%%%%%%%%%%%%%%%%%%%

We report for sake of completeness the general
expression of the generic order in the power series
expansion (\ref{sviluppo}). From the explicit calculation
in Lemma \ref{espansione} we get
\begin{multline*}
\langle q^{2}_{2n}\rangle\langle\phi_{s}\rangle
-\langle q^{2}_{2n+2}\rangle\langle\phi_{s}\rangle=\\
\sum_{m=2n-s+1}^{2n}\sum_{l=0}^{s-1}\sum_{a_1<\cdots<a_l}^{2,s}
(-)^m\binom{s+m+1}{m}\times \\
\mathbb{E}[w^m\Omega(\phi_{s}
\sigma^{a_1}_{i_1}\cdots\sigma^{a_l}_{i_1}
\sigma^{a_1}_{j_1}\cdots\sigma^{a_l}_{j_1})]\delta_{2n,m+l}\\
+\sum_{m=2n-s+2}^{2n+1}\sum_{l=0}^{s-1}\sum_{a_1<\cdots<a_l}^{2,s}
(-)^m\binom{s+m+1}{m}\times \\
\mathbb{E}[w^m\Omega(\phi_{s}\sigma^1_{i_1}\sigma^1_{j_1}
\sigma^{a_1}_{i_1}\cdots\sigma^{a_l}_{i_1}
\sigma^{a_1}_{j_1}\cdots\sigma^{a_l}_{j_1})]\delta_{2n,m+l-1}
\end{multline*}
which becomes, denoting by $x\land y$ the minumum between $x$ and $y$,
\begin{multline}\label{recursive}
\langle q^{2}_{2n}\rangle\langle\phi_{s}\rangle
-\langle q^{2}_{2n+2}\rangle\langle\phi_{s}\rangle=
\sum_{l=0}^{2n \land s-1}\sum_{a_1<\cdots<a_l}^{2,s}
(-)^{2n-l}\binom{2n+s-l+1}{2n-l}\times \\
[\langle \phi_{s}
q^{2}_{a_1\cdots a_{l}}q^{2}_{s+1\cdots s+2n-l}\rangle
-\frac{2n-l+s+2}{2n-l+1}
\langle\phi_{s} q^{2}_{1a_1\cdots a_{l}}q^{2}_{s+1\cdots s+2n-l+1}
\rangle]\  .
\end{multline}
In both the expressions above the term for $l=0$ is understood to be one.

The right hand side of (\ref{recursive}), due to the presence of 
$1+Jt^{-1}\sigma$ in the 
right hand side of (\ref{sviluppo}) - along with the 
symmetry of $J$, makes the expansion
somewhat recursive. This means that at each order we find some terms
already found in the previous order. More precisely, we claim
without proving that at each $2n$-th order 
of the expansion, all the terms involving $2m$-overlaps with $2m\leq 2n$
cancel out thanks to a repeated use of (\ref{sigma}) with 
the relations coming from the lower orders. Hence
from the $2n$-th order we get new relations involving $2n+2$-overlaps
only. This is what we explicitly verified only for 4-overlaps
in the previous pages.
More explicitly, if we re-write the difference in the right hand side  
of (\ref{recursive}) as
$$
\langle q^{2}_{2n}\rangle\langle\phi_{s}\rangle
-\langle q^{2}_{2n+2}\rangle\langle\phi_{s}\rangle=
c_{2n}-d_{2n+2}\ ,
$$
we have
$$
\langle q^{2}_{2n}\rangle\langle\phi_{s}\rangle=c_{2n}\ ,\ 
\langle q^{2}_{2n+2}\rangle\langle\phi_{s}\rangle=d_{2n+2}\ ,\
c_{2n}=d_{2n}\ .
$$
So that the final formula becomes
\begin{multline*}
\langle q^{2}_{2n}\rangle\langle\phi_{s}\rangle=\\
\sum_{l=0}^{2n \land s-1}\sum_{a_1<\cdots<a_l}^{2,s}
(-)^{2n-l}\binom{2n+s-l+1}{2n-l}
\langle q^{2}_{a_1\cdots a_{l}}q^{2}_{s+1\cdots s+2n-l}
\phi_{s}\rangle \ ,
\end{multline*}
and holds in integral average, as always.

%%%%%%%%%%%%%%%%%%%%%%%%%%%%%%%%%%%%%%
%%%%%%%%%%%%%%%%%%%%%%%%%%%%%%%%%%%%%%

\subsection{generalization to smooth functions of 
multi-overlaps}

Just like for the family of identities discussed in the previous section,
we started our analysis with the most natural quantity: the energy
of our model with 2-spin interactions.
And so we got again some relations for the squared multi-overlaps.
But we already know how to generalize these formulas to 
smooth functions of the overlaps. We can consider $p$-spin interactions,
and the procedure would provide us with the same relations for the 
$p$-th power of the overlaps. Then, as already explained, we can
take a convergent sum over all integer $p$ of $p$-spin Hamiltonians,
and consider the self-averaging of the desired one among them.
The perturbed Hamiltonian is again
$$
H_{N}(\sigma , \alpha ; \mathcal{J})=-\sum_{p}\bigg[a_{p}
\sum_{\nu=1}^{P^{(p)}_{\alpha N}}J_{\nu}\sigma_{i^{1}_{\nu}}
\cdots\sigma_{i^{p}_{\nu}}
+b_{p}\lambda_{p}\sum_{\nu=1}^{P^{\prime (p)}_{\alpha^{\prime}}}
J^{\prime}_{\nu}
\sigma_{j^{1}_{\nu}}
\cdots\sigma_{j^{p}_{\nu}}\bigg]\ ,
$$
where $\sum_{p}|a_{p}|^{2}=\sum_{p}|b_{p}|^{2}=1$, 
the notation for all the quenched variables is the usual one,
and $\{\lambda_{p}\}$ are the independent perturbing
real parameters. As a side remark, we just point out that
(like in \cite{gg}), in the case of this second family
of identities it is not necessary to consider a 
Hamiltonian consisting of the sum of all possible
$p$-spin Hamiltonians: only the perturbation must be
so.

Let us remind once again that all the identities we provided hold in integral
average with respect to all the variables used to compute derivatives.
If more that one variables is used in some derivatives, the relations hold
under integral performed simultaneously against all the variables, over
the Cartesian product of the chosen intervals of variation of each variable.


%%%%%%%%%%%%%%%%%%%%%%%%%%%%%%%%%%%%%%
%%%%%%%%%%%%%%%%%%%%%%%%%%%%%%%%%%%%%%


\section*{Concluding remarks}

Notice that while we derived our identities having as reference 
diluted spin glasses, all that matters in the derivation are the
properties of the perturbing Hamiltonian, and they are therefore
generically valid (in integral average with respect to the perturbing
parameters).

The Ghirlanda-Guerra identities for the overlap have been useful to
prove non trivial properties of mean-field spin glasses. For instance
Talagrand could prove \cite{talabook} that for all models 
where the identities are
valid, the overlap is positive with probability one.
This positivity property is important as it enters in the
Guerra free-energy bounds in spin systems without spin reversal
symmetry, like in the case of odd-spin interactions.  
Unfortunately the derivation
of Talagrand for the overlap does not extend immediately to the
multi-overlap case. We believe however that the self-averaging
identity will be useful in the mathematical analysis of diluted spin
models.




%%%%%%%%%%%%%%%%%%%%%%%%%%%%%%%%%%%%%%%%%%%%%%%%%%%%%%%
\section*{Acknowledgments}

LDS thanks Fabio Lucio Toninelli and Anton Bovier for useful discussions.




%%%%%%%%%%%%%%%%%%%%%%%%%%%%%%%%%%%%%%%%%%%%%%%%%%%%%%%

\begin{thebibliography}{99}

\bibitem{abds} A. Agostini, A. Barra, L. De Sanctis, {\em
    Positive-Overlap Transition and Critical Exponents in Mean Field
    Spin Glasses}, J. Stat. Mech. (2006) {\bf P11015}.

\bibitem{ac} M. Aizenman, P. Contucci, {\em On the stability of the
    quenched state in mean field spin glass models}, J. Stat. Phys.
  {\bf 92}, 765 (1998).

\bibitem{bds2} A. Barra, L. De Sanctis, {\em Stability properties and
    probability distribution of multi-overlaps in dilute spin
    glasses}, J. Stat. Mech. (2007) {\bf P08025}.

\bibitem{bovierbook} A. Bovier, {\em Statistical Mechanics of
    Disordered Systems, A mathematical perspective}, Cambridge
  University Press (2006).

\bibitem{bov2} A. Bovier, V. Gayrard, {\em Hopfield Models as
    Generalized Random Mean Field Models}, Arxiv:cond-mat/9607103
  (1997), in A. Bovier and P. Picco Eds., {\em Mathematical Aspects of
    Spin Glasses and Neural Networks}, Progress in Probability {\bf
    41} Birk\"auser, Boston-Basel-Berlin (1998).
  
\bibitem{lds1} L. De Sanctis, {\em Random Multi-Overlap Structures and
    Cavity Fields in Diluted Spin Glasses}, J. Stat. Phys.
  \textbf{117} 785-799 (2004)

\bibitem{flt} S. Franz, M. Leone, F.L. Toninelli, {\em Replica bounds for 
    diluted non-Poissonian spin systems}, J. Phys. A {\bf 36} 
  10967 (2003).
  
\bibitem{gg} S. Ghirlanda, F. Guerra, {\em General properties of overlap 
    distributions in disordered spin systems. Towards Parisi ultrametricity},
  J. Phys. A, {\bf 31} 9149-9155 (1998).
  
\bibitem{guerra2} F. Guerra, {\em About the overlap distribution in
    mean field spin glass models}, Int. Jou. Mod. Phys. B {\bf 10},
  1675-1684 (1996).
  
\bibitem{gt1} F. Guerra, F.L. Toninelli, {\em The high temperature region
    of the Viana-Bray diluted spin glass model}, 
  J. Stat. Phys.� \textbf{115}  (2004).
  
\bibitem{mm1}  M. M\'ezard, A. Montanari, {\em Reconstruction 
    on trees and spin glass transition}, ArXiv:cond-mat/0512295.

\bibitem{mpv} M. M\'ezard, G. Parisi and M. A. Virasoro, {\em Spin
    glass theory and beyond}, World Scientific, Singapore (1987).
  
\bibitem{mpz} M. M\'ezard, G. Parisi and R. Zecchina, {\em 
    Analytic and Algorithmic Solution of Random Satisfiability
    Problems}, Science {\bf 247} 812-815 (2002).
  
\bibitem{parisi} G. Parisi, {\em On the probabilistic formulation
    of the replica approach to spin glasses}, ArXiv:cond-mat/9801081.
  
\bibitem{t5} M. Talagrand, {\em The Sherrington Kirkpatrick model: a
    challenge for mathematicians}, Probab. Rel. Fields {\bf 110},
  109-176 (1998).
  
\bibitem{talabook} M. Talagrand, {\em Spin glasses: a challenge for
mathematicians. Cavity and Mean field models}, Springer Verlag
(2003).

\end{thebibliography}

\end{document}     


